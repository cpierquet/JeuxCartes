% !TeX TXS-program:compile = txs:///arara
% arara: pdflatex: {shell: yes, synctex: no, interaction: batchmode}
% arara: pdflatex: {shell: yes, synctex: no, interaction: batchmode} if found('log', '(undefined references|Please rerun|Rerun to get)')

\documentclass{article}
\usepackage[french]{babel}
\usepackage{iftex,ifpdf,ifxetex,ifluatex}
\ifpdftex
	\usepackage[utf8]{inputenc}
	\usepackage[T1]{fontenc}
	\usepackage[upright]{fourier}
	\usepackage[scaled=0.875]{helvet}
	\renewcommand\ttdefault{lmtt}
	\usepackage[scaled=0.875]{cabin}
	\usepackage{amsmath,amssymb}
\fi
\ifluatex
	\RequirePackage[math-style=french,bold-style=ISO]{fourier-otf}
\fi
\ifxetex
	\RequirePackage[math-style=french,bold-style=ISO]{fourier-otf}
\fi
\usepackage{fontawesome5}
\usepackage{enumitem}
\usepackage{JeuxCartes}
\usepackage{tabularray}
\usepackage{fancyvrb}
\usepackage{fancyhdr}
\fancyhf{}
\renewcommand{\headrulewidth}{0pt}
\lfoot{\sffamily \small [JeuxCartes]}
\cfoot{\sffamily \small - \thepage{} -}
\rfoot{\hyperlink{matoc}{\small\faArrowAltCircleUp[regular]}}

\usepackage{hvlogos}
\usepackage{menukeys}
\let\tab\relax
\usepackage{tabto}
\usepackage{codehigh}
\usepackage{scontents}
\usepackage{hyperref}
\urlstyle{same}
\hypersetup{pdfborder=0 0 0}
\usepackage[margin=1.5cm]{geometry}
\usepackage{newverbs}
\newverbcommand{\pverb}{\color{purple}}{}
\newverbcommand{\rverb}{\color{red}}{}
\newverbcommand{\vverb}{\color{ForestGreen}}{}
\newverbcommand{\averb}{\color{CadetBlue}}{}
\newverbcommand{\overb}{\color{orange}}{}
\newverbcommand{\bverb}{\color{blue}}{}
\setlength{\parindent}{0pt}
\definecolor{LightGray}{gray}{0.9}

\def\TPversion{0.1.2}
\def\TPdate{26 Octobre 2022}

\usepackage[most]{tcolorbox}
\usepackage[outputdir=build]{minted}
\tcbuselibrary{minted}

\tcbset{vignettes/.style={%
		nobeforeafter,box align=base,boxsep=0pt,enhanced,sharp corners=all,rounded corners=southeast,%
		boxrule=0.75pt,left=7pt,right=1pt,top=0pt,bottom=0.25pt,%
	}
}
\tcbset{vignettelatex/.style={%
		fontupper={\vphantom{pf}\footnotesize\ttfamily},
		vignettes,%
		colframe=CadetBlue,coltitle=white,colback=CadetBlue!5,%
		overlay={\begin{tcbclipinterior}%
				\fill[fill=lightgray!50]($(interior.south west)$) rectangle node[rotate=90]{\tiny \sffamily{\textcolor{CadetBlue}{\scalebox{0.6}[0.75]{\textbf{\LaTeX}}}}} ($(interior.north west)+(5pt,0pt)$);%
		\end{tcbclipinterior}}
	}
}

\newtcblisting{codetex}[1][]{%
	colback=white,colframe=red!75!black,title={\small \faCode} Code \LaTeX,fonttitle=\sffamily\bfseries,left=3pt,right=3pt,top=2pt,bottom=2pt,#1}

\newtcolorbox{codeattention}[1][]{%
	colback=Yellow!50,colframe=Yellow!50!Black,title={\small \faBomb} Attention,fonttitle=\sffamily\bfseries,left=3pt,right=3pt,top=2pt,bottom=2pt,#1}

\newtcolorbox{codesortie}[1][]{%
	colback=white,colframe=red!75!black,title={\small \faArrowAltCircleRight[regular]} Sortie \LaTeX,fonttitle=\sffamily\bfseries,left=3pt,right=3pt,top=2pt,bottom=2pt,#1}

\newtcolorbox{codeidee}[1][]{%
	colback=white,colframe=PeachPuff!75!black,title={\small \faLightbulb[regular]} Idée(s),fonttitle=\sffamily\bfseries,left=3pt,right=3pt,top=2pt,bottom=2pt,#1}

\newtcolorbox{codeinfo}[1][]{%
	colback=white,colframe=SteelBlue,title={\small \faPuzzlePiece} Information(s),fonttitle=\sffamily\bfseries,left=3pt,right=3pt,top=2pt,bottom=2pt,#1}

\newtcolorbox{codecles}[1][]{%
	colback=white,colframe=ForestGreen!75,title={\small \faPaperclip} Clés et options,fonttitle=\sffamily\bfseries,left=3pt,right=3pt,top=2pt,bottom=2pt,#1}

%petite vignette tex
\newcommand\ctex[1]{\tcbox[vignettelatex]{#1}}

\title{%
\begin{minipage}{0.85\linewidth}
	\begin{tcolorbox}[colframe=yellow,colback=yellow!15]
		\begin{center}
			\begin{tabular}{c}
				\lstinline!JeuxCartes!\\
				\\
				Des cartes de poker ou de tarot, \\
				simples ou en \textit{mains}, \\
				avec possibilité de tirage aléatoire.
			\end{tabular}
		\end{center}
	\end{tcolorbox}
\end{minipage}
}
\author{
	\begin{tabular}{c}
		Cédric Pierquet\\
		{\ttfamily c pierquet -- at -- outlook . fr}
	\end{tabular}
}
\date{Version \TPversion{} -- \TPdate}

\newcommand\Cle[1]{{\bfseries\sffamily\textlangle #1\textrangle}}

\begin{document}

\pagestyle{fancy}

\maketitle

\thispagestyle{empty}

{\Large {\bfseries Résumé} : Quelques commandes pour afficher des cartes à jouer, de type Poker ou Tarot}

\medskip

\noindent Une commande pour créer une carte individuelle (insertion autonome ou via \TikZ).

Une commande pour créer une main de cartes, avec possibilité d'affichage en éventail.

Une commande pour créer une main aléatoire de cartes, avec possibilité d'affichage en éventail.

Une commande pour des cartes en version \og miniatures \fg{} (individuelle, main ou aléatoire).

\vspace{0.5cm}

\begin{center}
	\CartesJeu[Eventail,EspH=0,EspV=0.1]{7K § 8P § DT § AP § 10C § AC § 8T § 2K § VC}\hspace{0.5cm}\CartesJeu[TypeJeu=Tarot,EspH=0.85,EspV=0.2]{Exc § 1AT § CC § 8T § 2K § 5AT § 2AT § DP § 7T § 10C § 19AT § VP}
\end{center}

\hfill{}\textit{Merci aux membres du groupe \faFacebook{} du \og Coin \LaTeX{} \fg{} pour leur aide et leurs idées !}

\vfill

\hrule

\medskip

\begin{tblr}{width=\linewidth,colspec={X[c]X[c]X[c]X[c]X[c]X[c]},cells={font=\sffamily}}
	{\huge \LaTeX} & & & & &\\
	& {\huge \pdfLaTeX} & & & & \\
	& & {\huge \LuaLaTeX} & & & \\
	& & & {\huge \TikZ} & & \\
	& & & & {\huge \TeXLive} & \\
	& & & & & {\huge \MiKTeX} \\
\end{tblr}

\medskip

\hrule

\vfill

~

\newpage

\phantomsection
\hypertarget{matoc}{}

\tableofcontents

\vfill~

\part{Historique}
{\small \bverb|v0.1.2|~:~~~~Modification du nom (et de la source) des images de Tarot (CC0)

{\small \bverb|v0.1.1|~:~~~~Ajout de commandes pour des mini-cartes

{\small \bverb|v0.1  |~:~~~~Version initiale

\vfill~

\newpage

\part*{Introduction}

\section{Le package JeuxCartes}

\subsection{\og Philosophie \fg{} du package}

\begin{codeidee}
L'idée de ce package est de proposer, comme le fait \ctex{pst-poker}, des commandes pour intégrer des cartes de poker dans un document \LaTeX.

Les cartes sont des images :
%
\begin{itemize}
	\item les cartes type \textsf{Poker} sont sous licence LGPL, \copyright{}2005, de David Bellot, via \url{https://github.com/htdebeer/SVG-cards} ;
	\item les cartes type \textsf{Tarot} sous licence CC0, du site \url{https://freesvg.org}.
\end{itemize}
\end{codeidee}

\begin{codeinfo}
Le {package} \ctex{JeuxCartes} charge les {packages} :

\begin{itemize}
	\item \ctex{tikz}, \ctex{pifont}, \ctex{xfp}, \ctex{pgffor}, \ctex{xinttools} ;
	\item \ctex{listofitems}, \ctex{xstring}, \ctex{xparse} et \ctex{simplekv} ;
	\item \ctex{randomlist}, \ctex{xcolor} avec les options \ctex{[table,svgnames]}.
\end{itemize}
\end{codeinfo}

\begin{codeinfo}
Des packages ou solutions existent déjà pour des cartes en \LaTeX, le lecteur choirira donc la solution qui lui semblera être la meilleure pour son utilisation !

On peut citer par exemple :

\begin{itemize}
	\item le packages \ctex{pst-poker} en language \texttt{pstricks} ;
	\item les fichiers \texttt{postscript} disponibles sur \url{https://melusine.eu.org/syracuse/postscript/cartes01/}
\end{itemize}
\end{codeinfo}

\begin{codeattention}
Les images utiles sont proposés en format \texttt{png}, donc une solution de compilation adaptée au format \textsf{png} est nécessaire.
\end{codeattention}

\subsection{Chargement du package}

\begin{codetex}[listing only]
%exemple de chargement pour une compilation en (pdf)latex
\documentclass{article}
\usepackage[french]{babel}
\usepackage[utf8]{inputenc}
\usepackage[T1]{fontenc}
\usepackage{JeuxCartes}
...
\end{codetex}

\begin{codetex}[listing only]
%exemple de chargement pour une compilation en (xe/lua)latex
\documentclass{article}
\usepackage[french]{babel}
\usepackage{mathtools}
\usepackage{fontspec}
\usepackage{JeuxCartes}
...
\end{codetex}

\section{Compléments}

\subsection{Le système de \og clés/options \fg}

\begin{codeidee}
L'idée est de conserver -- autant que faire se peut -- l'idée de \Cle{Clés} qui sont modifiables et relativement explicites. À noter que :
%
\begin{itemize}[leftmargin=*]
	\item les \Cle{Clés} peuvent être mises dans n'importe quel ordre, et être omises lorsque la valeur par défaut est conservée ;
	\item les \textsf{arguments} doivent, eux, être positionnés dans le \textit{bon ordre}.
\end{itemize}
\end{codeidee}

\begin{codeinfo}
Les \textsf{commandes} présentés seront explicités via leur \textsf{syntaxe} avec les \textsf{options} ou \textsf{arguments}.

Autant que faire se peut, des exemples/illustrations/remarques seront proposés à chaque fois.

\smallskip

Les \textsf{codes} seront présentés dans des \textsf{boîtes} \textcolor{red!75!black}{{\small \faCode} Code \LaTeX}, si possible avec la \textsf{sortie} dans la même boîte, et sinon la \textsf{sortie} sera visible dans des \textsf{boîtes} \textcolor{red!75!black}{{\small \faArrowAltCircleRight[regular]} Sortie \LaTeX}. Les \textsf{clés} ou \textsf{options} seront présentées dans des \textsf{boîtes} \textcolor{ForestGreen}{{\small \faPaperclip} Clés}.
\end{codeinfo}

\part{Les commandes}

\section{La commande simple \og CarteJeu \fg}

\subsection{Introduction}

\begin{codeinfo}
La commande \ctex{AffCarteJeu} affiche une carte, avec un système de clés/options
\end{codeinfo}

\begin{codetex}[]
\AffCarteJeu{VP}\AffCarteJeu{10K}\AffCarteJeu{AC}\AffCarteJeu{DosBleu}\AffCarteJeu{DT}\\ \\
\AffCarteJeu[TypeJeu=Tarot]{VP}\AffCarteJeu[TypeJeu=Tarot]{Dos}
\AffCarteJeu[TypeJeu=Tarot]{Exc}\AffCarteJeu[TypeJeu=Tarot]{1AT}
\AffCarteJeu[TypeJeu=Tarot]{10K}\AffCarteJeu[TypeJeu=Tarot]{18AT}
\end{codetex}

\subsection{Noms de cartes}

\begin{codeinfo}
Les cartes étant des \textbf{images}, leur nom 
est fixé suivant le modèle suivant (typiquement \texttt{<hauteur>.<couleur>} ) avec quelques spécificités :

\begin{itemize}
	\item cartes classiques pour Belote/Bataille/Poker :
	\begin{itemize}
		\item \texttt{2C}, \texttt{3C}, \ldots, \texttt{DC}, \texttt{RC}, \texttt{AC}\dotfill\Cle{Cœur}
		\item \texttt{2P}, \texttt{3P}, \ldots, \texttt{DP}, \texttt{RP}, \texttt{AP}\dotfill\Cle{Pique}
		\item \texttt{2K}, \texttt{3K}, \ldots, \texttt{DK}, \texttt{RK}, \texttt{AK}\dotfill\Cle{Carreau}
		\item \texttt{2T}, \texttt{3T}, \ldots, \texttt{DT}, \texttt{RT}, \texttt{AT}\dotfill\Cle{Trèfle}
		\item \texttt{JN}, \texttt{JR}\dotfill\Cle{Joker}
		\item \texttt{DosRouge}, \texttt{DosBleu}, \ldots\dotfill\Cle{Dos}
	\end{itemize}
	\item cartes de Tarot  :
	\begin{itemize}
		\item \texttt{2C}, \texttt{3C}, \ldots, \texttt{CC}, \ldots, \texttt{AC}\dotfill\Cle{Cœur}
		\item \texttt{2P}, \texttt{3P}, \ldots, \texttt{CP}, \ldots, \texttt{AP}\dotfill\Cle{Pique}
		\item \texttt{2K}, \texttt{3K}, \ldots, \texttt{CK}, \ldots, \texttt{AK}\dotfill\Cle{Carreau}
		\item \texttt{2T}, \texttt{3T}, \ldots, \texttt{CT}, \ldots, \texttt{AT}\dotfill\Cle{Trèfle}
		\item \texttt{Exc}, \texttt{1AT}, \texttt{2AT}, \ldots, \texttt{20T}, \texttt{21AT}\dotfill\Cle{Atouts}
		\item \texttt{Dos}\dotfill\Cle{Dos}
	\end{itemize}
\end{itemize}
\end{codeinfo}

\subsection{Clés et options}

\begin{codetex}[listing only]
\AffCarteJeu[options]{<carte>}
\end{codetex}

\begin{codecles}
Concernant les \Cle{options} :

\begin{itemize}
	\item la clé \Cle{Hauteur} correspond à la hauteur voulue, en cm, de la carte ; \hfill{}défaut \Cle{4.25}
	\item la clé \Cle{TypeJeu} (parmi \Cle{Poker} ou \Cle{Tarot}) pour spécifier les cartes à utiliser ; \hfill{}défaut \Cle{Poker}
	\item la clé \Cle{Rotation} pour une éventuelle rotation de la carte ; \hfill{}défaut \Cle{0}
	\item la clé \Cle{AlignementV} pour l'alignement vertical (lié à \ctex{raisebox} ou à \TikZ) ; \hfill{}défaut \Cle{0.5}
	\item le booléen \Cle{Tikz} pour préciser que la carte est \textit{autonome}, dans un environnement \ctex{tikzpicture} ; \hfill{}défaut \Cle{false}
	\item la clé \Cle{DecalageX} pour le décalage horizontal par rapport à l'origine (si \Cle{Tikz=false}) ; \hfill{}défaut \Cle{0}
	\item la clé \Cle{DecalageY} pour le décalage vertical par rapport à l'origine (si \Cle{Tikz=false}) ; \hfill{}défaut \Cle{0}
	\item le booléen \Cle{TikzAutonome} pour basculer d'un environnement \TikZ{} à une commande \TikZ. \hfill{}défaut \Cle{true}
\end{itemize}
\end{codecles}

\begin{codeinfo}
Concernant la clé \Cle{AlignementV}, elle peut être  :

\begin{itemize}
	\item donnée sous forme décimale entre 0 et 1 pour l'utilisation standard ;
	\item choisie parmi 0, 0.5 et 1 pour l'utilisation via le mode \Cle{Tikz}.
\end{itemize}
\end{codeinfo}

\begin{codeinfo}
Concernant la clé \Cle{TikzAutonome} est plutôt une clé utilisée \textit{interne}, elle peut être utilisée pour intégrer une carte dans un environnement \TikZ{} déjà créé, pour une utilisation complètement personnalisé des cartes !

\smallskip

Il est à noter que la commande (en \textit{interne}) est liée à un \textit{sous}-environnement \ctex{scope}.
\end{codeinfo}


\begin{codetex}[]
En mode normal, \AffCarteJeu{VP} ou \AffCarteJeu[AlignementV=0]{DosRouge} ou
\AffCarteJeu[AlignementV=1]{10K} ou \AffCarteJeu[TypeJeu=Tarot,AlignementV=0.33,Rotation=25]{10AT}.

\smallskip

En mode \TikZ, \AffCarteJeu[Hauteur=2.75,Tikz]{VP}
ou \AffCarteJeu[Hauteur=2.75,AlignementV=0,Tikz]{DosVert}
ou \AffCarteJeu[Hauteur=2.75,AlignementV=1,Tikz]{10K}
ou \AffCarteJeu[Hauteur=2.75,TypeJeu=Tarot,AlignementV=0.33,Rotation=-25]{10AT}.

\smallskip

On obtient la main \AffCarteJeu[Hauteur=2]{VP}\AffCarteJeu[Hauteur=2]{10K}
\AffCarteJeu[Hauteur=2]{AK}\AffCarteJeu[Hauteur=2]{JN}\AffCarteJeu[Hauteur=2]{7P}.
\end{codetex}

\section{Main de cartes}

\subsection{Introduction}

\begin{codeidee}
La commande précédente, relativement simple, ne fait \textit{que} placer une image dans un le document, il sera peut-être plus intéressant d'utiliser les commandes qui suivent, afin de créer des \textit{mains} de cartes, aléatoires ou non !
\end{codeidee}

\begin{codeattention}
Les commandes pour créer des \textit{mains} sont obligatoirement liées à des environnement \ctex{tikzpicture} autonomes !
\end{codeattention}

\begin{codetex}[listing only]
\CartesJeu[options]{<liste de cartes>}
\end{codetex}

\subsection{Clés et options}

\begin{codecles}
L'argument \textit{mandataire} correspond à la liste des cartes à afficher, sous la forme 

\begin{itemize}
	\item \texttt{<carte1> § <carte2> § ... § <carten>}.
\end{itemize}

Les \Cle{options} sont les suivantes :

\begin{itemize}
	\item la clé \Cle{Hauteur} correspond à la hauteur voulue, en cm, des cartes ; \hfill{}défaut \Cle{4.25}
	\item la clé \Cle{TypeJeu} (parmi \Cle{Poker} ou \Cle{Tarot}) pour spécifier les cartes à utiliser ; \hfill{}défaut \Cle{Poker}
	\item la clé \Cle{EspH} correspond à coefficient d'espacement horizontal (\textit{proche} du cm en taille réelle) entre les cartes ; \hfill{}défaut \Cle{1}
	\item la clé \Cle{EspH} correspond à coefficient d'espacement vertical (\textit{proche} du cm en taille réelle) entre les cartes ; \hfill{}défaut \Cle{0}
	\item le booléen \Cle{Eventail} pour une présentation en éventail ; \hfill{}défaut \Cle{false}
	\item la clé \Cle{Rotation} pour l'angle entre les cartes en mode \Cle{Eventail}  ; \hfill{}défaut \Cle{10}
	\item la clé \Cle{AlignementV} pour l'alignement vertical lié à \TikZ{}\hfill{}défaut \Cle{0.5}
\end{itemize}
\end{codecles}

\begin{codetex}[]
\CartesJeu{7K § 8P § DT § AC}
~ou \CartesJeu[Eventail,EspH=0,EspV=0.1]{7K § 8P § DT § AC}
~ou \CartesJeu[Eventail,EspH=0.5]{7K § 8P § DT § AC}
\end{codetex}

\begin{codeinfo}
Pour une main \textit{horizontale}, seule la clé \Cle{EspH} est importante, et une valeur proche de 1 donne des résultats intéressants.

\smallskip

Pour une main \textit{éventail}, les deux espacements peuvent être modifiés, afin d'avoir un rendu satisfaisant pour l'utilisateur.

\smallskip

Les exemples proposés dans cette documentation permettent de se rendre compte de valeurs possibles pour un rendu satisfaisant.
\end{codeinfo}

\begin{codetex}[]
\CartesJeu[TypeJeu=Tarot]%
	{Exc § 1AT § CC § 8T § 2K § 5AT § 2AT § DP § 7T § 10C § 19AT § VP}

\smallskip

\CartesJeu[TypeJeu=Tarot,EspH=1,EspV=-0.15,Hauteur=2.25]%
	{Exc § 1AT § CC § 8T § 2K § 5AT § 2AT § DP § 7T § 10C § 19AT § VP}

\smallskip

Ça c'est une belle poignée ! \CartesJeu[Eventail,Hauteur=3,TypeJeu=Tarot,EspH=0,EspV=0.1]%
	{Exc § 1AT § 2AT § 3AT § 4AT § 5AT § 6AT § %
	10AT § 11AT § 15AT § 16AT § 19AT § 20AT § 21AT}

\smallskip

Et un chien qui en a  : \CartesJeu[Eventail,Hauteur=2,TypeJeu=Tarot,EspH=0.5,EspV=0.15,Rotation=20]%
	{DK § 10AT § 16AT § VP § 1AT}
\end{codetex}

\pagebreak

\section{Mains aléatoires}

\subsection{Introduction}

\begin{codeidee}
L'idée est ici de proposer une commande, similaire à la précédente, mais qui permet de \textbf{tirer au hasard} un certain nombre de cartes pour créer une main.
\end{codeidee}

\begin{codetex}[listing only]
\CartesJeuAleatoire[<options>]{<nombre de cartes>}
\end{codetex}

\begin{codetex}[]
Voilà de quoi d'obtenir une main aléatoire de Poker :

\CartesJeuAleatoire{5}
\end{codetex}

\subsection{Clés et options}

\begin{codecles}
Les \Cle{options} sont les suivantes :

\begin{itemize}
	\item la clé \Cle{Hauteur} correspond à la hauteur voulue, en cm, des cartes ; \hfill{}défaut \Cle{4.25}
	\item la clé \Cle{TypeJeu} (parmi \Cle{Poker} ou \Cle{Tarot} ou \Cle{Belote} ou \Cle{Bataille}) ; \hfill{}défaut \Cle{Poker}
	\item la clé \Cle{EspH} correspond à coefficient d'espacement horizontal (\textit{proche} du cm en taille réelle) entre les cartes ; \hfill{}défaut \Cle{1}
	\item la clé \Cle{EspH} correspond à coefficient d'espacement vertical (\textit{proche} du cm en taille réelle) entre les cartes ; \hfill{}défaut \Cle{0}
	\item le booléen \Cle{Eventail} pour une présentation en éventail ; \hfill{}défaut \Cle{false}
	\item la clé \Cle{Rotation} pour l'angle entre les cartes en mode \Cle{Eventail}  ; \hfill{}défaut \Cle{10}
	\item la clé \Cle{AlignementV} pour l'alignement vertical lié à \TikZ{}\hfill{}défaut \Cle{0.5}
\end{itemize}
\end{codecles}

\begin{codeinfo}
En ce qui concerne les \textit{jeux} disponibles (non modifiables) :

\begin{itemize}
	\item \Cle{Poker} : 52 cartes (sans Joker) ;
	\item \Cle{Belote} : 32 cartes (sans Joker) ;
	\item \Cle{Bataille} : 54 cartes (avec Jokers) ;
	\item \Cle{Tarot} : 78 cartes.
\end{itemize}
\end{codeinfo}

\pagebreak

\begin{codetex}[]
\textbf{\large Belote :} \\

\CartesJeuAleatoire[Hauteur=2,TypeJeu=Belote]{8} ou \CartesJeuAleatoire[Hauteur=2,TypeJeu=Belote]{8}.\\

\textbf{\large Poker :} \\

\CartesJeuAleatoire[Hauteur=3,Eventail,EspH=0,EspV=0.1]{5} ou  \CartesJeuAleatoire[Hauteur=3,Eventail,EspH=0,EspV=0.1]{5}.\\

\textbf{\large Bataille :} \\

\CartesJeuAleatoire[Hauteur=2.25,TypeJeu=Bataille]{27}\\

\textbf{\large Tarot :} \\

\CartesJeuAleatoire[Hauteur=3.25,TypeJeu=Tarot]{10} ou~
\CartesJeuAleatoire[Hauteur=3.25,Eventail,TypeJeu=Tarot,EspH=0,EspV=0.1]{10}
\end{codetex}

\pagebreak

\section{\textit{Mini}-Cartes}

\subsection{Introduction}

\begin{codeidee}
L'idée est ici de proposer des commandes pour afficher des \textit{mini-}cartes, utilisables en mode ligne, sans utiliser les images précédentes.

\smallskip

Ces \textit{mini-}cartes sont des figures \TikZ, alignées verticalement sur leur \textit{baseline}.
\end{codeidee}

\begin{codetex}[listing only]
\MiniCarteJeu[<options>]{<carte>}
\end{codetex}

\begin{codetex}[]
Si on met du texte sur la ligne du dessus, on peut voir le résultat.\\
Voilà des exemples de mini-cartes, \MiniCarteJeu{7.K}\MiniCarteJeu{1.AT}\MiniCarteJeu{V.K}\MiniCarteJeu{10.C}, intégrables dans un paragraphe.\\
Si on met du texte sur la ligne du dessous, on peut voir le résultat.
\end{codetex}

\subsection{Noms des \textit{mini-}cartes}

\begin{codeinfo}
Pour des raisons internes au code, les cartes doivent être saisies suivant la nomenclature (noter l'utilisation du \ctex{.} pour séparer la hauteur de la couleur !)  :

\begin{itemize}
	\item \texttt{2.C}, \texttt{3.C}, \ldots, \texttt{C.C}, \texttt{D.C}, \texttt{R.C}, \texttt{A.C}\dotfill\Cle{Cœur}
	\item \texttt{2.P}, \texttt{3.P}, \ldots, \texttt{C.P}, \texttt{D.P}, \texttt{R.P}, \texttt{A.P}\dotfill\Cle{Pique}
	\item \texttt{2.K}, \texttt{3.K}, \ldots, \texttt{C.K}, \texttt{D.K}, \texttt{R.K}, \texttt{A.K}\dotfill\Cle{Carreau}
	\item \texttt{2.T}, \texttt{3.T}, \ldots, \texttt{K.T}, \texttt{D.T}, \texttt{R.T}, \texttt{A.T}\dotfill\Cle{Trèfle}
	\item \texttt{J.N}, \texttt{J.R}\dotfill\Cle{Joker}
	\item \texttt{Exc}, \texttt{1.AT}, \texttt{2.AT}, \ldots, \texttt{20.T}, \texttt{21.AT}\dotfill\Cle{Atouts}
\end{itemize}
\end{codeinfo}

\begin{codeinfo}
Les \textit{mini-}cartes d'atout pour le Tarot sont présentées avec un fond coloré, et avec un symbole en étoile.
\end{codeinfo}

\subsection{Clés et options}

\begin{codecles}
Quelques \Cle{options} pour les \textit{mini-}cartes :

\begin{itemize}
	\item la clé \Cle{Largeur} pour gérer la largeur des miniatures ; \hfill{}défaut \Cle{0.55cm}
	\item la clé \Cle{FondAtout} pour gérer la couleur de fond pour les atouts. \hfill{}défaut \Cle{PeachPuff}
\end{itemize}
\end{codecles}

\begin{codetex}[]
\foreach \EECARTE in {2,3,4,5,6,7,8,9,10,V,C,D,R,A}{\MiniCarteJeu[Largeur=0.75cm]{\EECARTE.K}}\\
\MiniCarteJeu[FondAtout=LightSkyBlue,Largeur=0.75cm]{Exc}%
\foreach \EECARTE in {1,2,...,21}{\MiniCarteJeu[FondAtout=LightSkyBlue,Largeur=0.75cm]{\EECARTE.AT}}
\end{codetex}

\subsection{\textit{Mini}-Mains et \textit{mini}-mains aléatoires}

\begin{codeidee}
Comme pour les cartes \textit{classiques}, il existe deux commandes pour des \textit{mains} de \textit{mini-}cartes, mais uniquement en présentation horizontale/côte-à-côte.
\end{codeidee}

\begin{codetex}[listing only]
\MiniCartesJeu[<options>]{<liste de cartes>}

\MiniCartesJeuAleatoire[<options>]{<liste de cartes>}
\end{codetex}

\begin{codecles}
Les \Cle{Clés} sont les mêmes que pour la commande individuelle, avec en plus :

\begin{itemize}
	\item la clé \Cle{TypeJeu} (parmi \Cle{Poker} ou \Cle{Tarot} ou \Cle{Belote} ou \Cle{Bataille}). \hfill{}défaut \Cle{Poker}
\end{itemize}
\end{codecles}

\begin{codetex}[]
\textbf{\large Saisie de mains : }\\
\MiniCartesJeu{7.K § A.P § D.T § V.K § 10.C § C.T} et~
\MiniCartesJeu{Exc § 1.AT § C.C § 8.T § 2.K § 5.AT § 2.AT § D.P § 7.T § 10.C § 19.AT § V.P}\\

\textbf{\large Poker : }\\
\MiniCartesJeuAleatoire{5} ou \MiniCartesJeuAleatoire{5} ou \MiniCartesJeuAleatoire{5} ou \MiniCartesJeuAleatoire{5}.\\

\textbf{\large Belote :}\\
\MiniCartesJeuAleatoire[TypeJeu=Belote]{8} ou \MiniCartesJeuAleatoire[TypeJeu=Belote]{8} ou \MiniCartesJeuAleatoire[TypeJeu=Belote]{8}.\\

\textbf{\large Bataille : }\\
\MiniCartesJeuAleatoire[TypeJeu=Bataille]{12} ou \MiniCartesJeuAleatoire[TypeJeu=Bataille]{12}.\\

\textbf{\large Tarot : }\\
\MiniCartesJeuAleatoire[TypeJeu=Tarot]{10} ou \MiniCartesJeuAleatoire[TypeJeu=Tarot]{10}.
\end{codetex}

\pagebreak

\part{Compléments}

\section{Cartes disponibles}

\subsection{Poker/Bataille/Belote}

\xintFor #1 in {2,3,4,5,6,7,8,9,10,V,D,R,A}\do{\AffCarteJeu[Hauteur=1.75]{#1P}}

\xintFor #1 in {2,3,4,5,6,7,8,9,10,V,D,R,A}\do{\AffCarteJeu[Hauteur=1.75]{#1C}}

\xintFor #1 in {2,3,4,5,6,7,8,9,10,V,D,R,A}\do{\AffCarteJeu[Hauteur=1.75]{#1K}}

\xintFor #1 in {2,3,4,5,6,7,8,9,10,V,D,R,A}\do{\AffCarteJeu[Hauteur=1.75]{#1T}}

\AffCarteJeu[Hauteur=1.75]{JN}\AffCarteJeu[Hauteur=1.75]{JR}%
\xintFor #1 in {Bleu,BleuVert,Fuchsia,Gris,Jaune,Marron,Navy,Noir,Olive,Vert,Violet}\do{\AffCarteJeu[Hauteur=1.75]{Dos#1}}

\subsection{Cartes Tarot}

\xintFor #1 in {2,3,4,5,6,7,8,9,10,V,C,D,R,A}\do{\AffCarteJeu[TypeJeu=Tarot,Hauteur=1.75]{#1P}}

\xintFor #1 in {2,3,4,5,6,7,8,9,10,V,C,D,R,A}\do{\AffCarteJeu[TypeJeu=Tarot,Hauteur=1.75]{#1C}}

\xintFor #1 in {2,3,4,5,6,7,8,9,10,V,C,D,R,A}\do{\AffCarteJeu[TypeJeu=Tarot,Hauteur=1.75]{#1K}}

\xintFor #1 in {2,3,4,5,6,7,8,9,10,V,C,D,R,A}\do{\AffCarteJeu[TypeJeu=Tarot,Hauteur=1.75]{#1T}}

\AffCarteJeu[TypeJeu=Tarot,Hauteur=1.75]{Exc}%
\xintFor #1 in {1,2,3,4,5,6,7,8,9,10}\do{\AffCarteJeu[TypeJeu=Tarot,Hauteur=1.75]{#1AT}}%
\AffCarteJeu[TypeJeu=Tarot,Hauteur=1.75]{Dos}

\xintFor #1 in {11,12,13,14,15,16,17,18,19,20,21}\do{\AffCarteJeu[TypeJeu=Tarot,Hauteur=1.75]{#1AT}}%
\AffCarteJeu[TypeJeu=Tarot,Hauteur=1.75]{Dos}

\pagebreak

\subsection{MiniCartes}

\xintFor #1 in {2,3,4,5,6,7,8,9,10,V,C,D,R,A}\do{\MiniCarteJeu{#1.P}}

\xintFor #1 in {2,3,4,5,6,7,8,9,10,V,C,D,R,A}\do{\MiniCarteJeu{#1.C}}

\xintFor #1 in {2,3,4,5,6,7,8,9,10,V,C,D,R,A}\do{\MiniCarteJeu{#1.K}}

\xintFor #1 in {2,3,4,5,6,7,8,9,10,V,C,D,R,A}\do{\MiniCarteJeu{#1.T}}

\MiniCarteJeu{J.N}\MiniCarteJeu{J.R}

\medskip

\MiniCarteJeu{Exc}%
\xintFor #1 in {1,2,3,4,5,6,7,8,9,10}\do{\MiniCarteJeu{#1.AT}}

\xintFor #1 in {11,12,13,14,15,16,17,18,19,20,21}\do{\MiniCarteJeu{#1.AT}}%

\section{Carte simple dans un environnement \TikZ}

\begin{codeidee}
La commande simple, avec l'option \Cle{TikzAutonome=false} permet à l'utilisateur de gérer les cartes dans un environnement \ctex{tikzpicture} \textit{indépendant}.

\smallskip

L'\textit{origine} des cartes est fixée, au coin bas-gauche, et la \Cle{Rotation} éventuelle est appliquée après la translation gérée par les \Cle{DecalageXY}.
\end{codeidee}

\begin{codetex}[]
\begin{tikzpicture}
	\draw[thin,lightgray] (-5,-5) grid (5,3) ; %grille rajoutée pour la lisibilité du code
	\foreach \ptsupport in {(0,0),(3,-4),(-3,-1),(-2,-3.5),(2,2.75)}%
		{\filldraw[Gray] \ptsupport circle[radius=2pt] ;}  %poits rajouté pour la lisibilité du code
	\foreach \x in {-5,-4,...,5} \draw[lightgray] (\x,-5) node[below,font=\small] {\x} ;
	\foreach \y in {-5,-4,...,3} \draw[lightgray] (-5,\y) node[left,font=\small] {\y} ;
	\AffCarteJeu[Hauteur=2,Tikz=true,TikzAutonome=false]%
		{10C}
	\AffCarteJeu[Hauteur=2.5,Tikz=true,TikzAutonome=false,DecalageX=3,DecalageY=-4]%
		{JN}
	\AffCarteJeu[Hauteur=3,Tikz=true,TikzAutonome=false,DecalageX=-3,DecalageY=-1,Rotation=20]%
		{JR}
	\AffCarteJeu[Hauteur=1.75,Tikz=true,TikzAutonome=false,DecalageX=-2,DecalageY=-3.5,Rotation=-50]%
		{DosBleu}
	\AffCarteJeu%
		[TypeJeu=Tarot,Hauteur=2.75,Tikz=true,TikzAutonome=false,DecalageX=2,DecalageY=2.75,Rotation=-90]%
		{18AT}
\end{tikzpicture}
\end{codetex}

\end{document}